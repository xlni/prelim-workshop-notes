\section{7/18/18: Group theory}
\begin{problem}
	Let $G$ be a finite group of order $n$ with the property that for each divisor $d$ of $n$ there is at most one subgroup in $G$ of order $d$. Show $G$ is cyclic.
\end{problem}
\begin{solution}
	First we reduce to the case that $|G|= p^\alpha$ for some prime $p$. Note that if
	\[
		|G| = p_1^{\alpha_1} \cdots p_k^{\alpha_k}
	\]
	then we can invoke the Sylow theorems to deduce that each Sylow $p$-subgroup is normal, and that $G$ is the direct product of them. So, if we show the result for $p^\alpha$, then the general case follows from the Chinese Remainder Theorem for instance.
	
	Next is essentially a counting argument. If $G$ has no elements of order $p^\alpha$, then it can have at most one subgroup of each of the orders $1, p, \ldots, p^{\alpha-1}$, and each element must belong to one of these subgroups. But
	\[
		1 + p + \cdots + p^{\alpha-1} = \frac{p^\alpha - 1}{p - 1} < p^\alpha
	\]
	so this is impossible.
\end{solution}

\begin{problem}\hfill
	\begin{enumerate}
		\item Let $G$ be a finite group and let $X$ be the set of pairs of commuting elements of $G$
		\[
			X = \{(g,h)\subset G\times G : gh=hg\}.
		\]
		Prove that $|X| = c|G|$ where $c$ is the number of conjugacy classes in $G$.
		\item Compute the number of pairs of commuting permutations on five letters.
	\end{enumerate}
\end{problem}
\begin{solution}\hfill
	\begin{enumerate}
		\item For each $g \in G$, we count the number of elements that commute with $g$. Consider the action of $G$ on itself by conjugation. The elements which commute with $g$ is then $\Stab(g)$. The orbit-stabilizer theorem tells us
		\[
			[G: \Stab(g)] = |C_g|
		\]
		where $C_g$ is the conjugacy class of $g$. Hence we sum $|G|/|C_g|$ over all elements $g\in G$. But the sum over each conjugacy class is $|G|$, so the total sum is $c|G|$ as claimed.
		\item The number of conjugacy classes in $S_5$ is the number of partitions of the number $5$: there are $7$ of these.
		Hence by the preceding part the answer is $5!7 = 840$.
	\end{enumerate}
\end{solution}

\begin{problem}
	Prove that every group of order 30 has a cyclic subgroup of order 15.
\end{problem}
\begin{solution}
	Intense Sylow bashing. Essentially there are two steps: first show that every group of order 30 has a subgroup of order 15, and then that the only group of order 15 is $\bb{Z}/15$.
	
	First step: let $H$ be a Sylow 3-subgroup and $K$ be a Sylow 5-subgroup. Use counting to show that one of these has no conjugates (it is the unique Sylow $p$-group). Then $HK$ is a subgroup of order 15.
	
	Second step: use Sylow theorems again to show that every group of order 15 has only one Sylow 3-group and one 5-group, so they are both normal and the group is the direct product $\bb{Z}/15$.
\end{solution}

\begin{problem}
	Find the smallest $n$ for which the permutation group $S_n$ contains a cyclic subgroup of order $111$.
\end{problem}
\begin{solution}
	Asking for a cyclic subgroup of order $111$ is the same as asking for an element of order $111$. Every element of $S_n$ has a disjoint cycle decomposition, and the order of an element is the LCM of the cycle lengths in this decomposition.
	
	Since $111 = 3 \cdot 37$, the answer is $3+37 = 40$. An element such as
	\[
		(1 \: 2 \: 3 \: \cdots \: 37)(38 \: 39 \: 40)
	\]
	will do the trick.
\end{solution}

\begin{problem}[Poincare's Theorem]
	Let $G$ be a group and $H \leq G$ a subgroup of finite index $n$. Show that $G$ contains a normal subgroup $N$ such that $N \leq H$ and the index of $N$ is $\leq n!$.
\end{problem}
\begin{solution}
	The presence of $n!$ strongly suggests that we do something with $S_n$. Consider the (left) action of $G$ on the (left) cosets of $H$. This action defines a map $\alpha \colon G \to S_n$. Note that the only elements of $G$ which fix the coset containing the identity $1$ are the elements of $H$. Thus $N\coloneqq \ker \alpha \subseteq H$. Since $G/N \cong \alpha(G)$, we have
	\[
		|G/N| = |\alpha(G)| \leq |S_n| = n!
	\]
	as wanted.
\end{solution}

\begin{problem}\hfill
	\begin{enumerate}
		\item Let $G$ be a non-abelian finite group. Show that $G/Z(G)$ is not cyclic, where $Z(G)$ is the center of $G$.
		\item If $|G| = p^n$, with $p$ prime and $n > 0$, show that $Z(G)$ is not trivial.
		\item If $|G| = p^2$, show that $G$ is abelian.
	\end{enumerate}
\end{problem}
\begin{solution}\hfill
	\begin{enumerate}
		\item This is easy to show by contradiction; suppose that $G/Z(G)$ is cyclic and generated by an element $[\alpha]$, where $\alpha \in G$. Then any element of $G$ can be written in the form $\alpha^k z$ where $z\in Z(G)$. But elements of this form commute, so $G$ must be abelian.
		\item Use the ``class equation,'' which is to say we consider partitioning $G$ into its conjugacy classes. The singleton conjugacy classes correspond to elements of $Z(G)$. Since the identity is in its own conjugacy class, there must be at least $p$ singletons, as the size of each conjugacy class must divide $|G| = p^n$.
		\item The center cannot be of size $p$ by the first part, and it cannot be trivial by the second. So it must be of size $p^2$, i.e. the whole group.
	\end{enumerate}
\end{solution}

\begin{problem}
	Use the simplicity of $A_6$ to show that $A_6$ does not have an index 3 subgroup. Then show that there are no simple groups of order 120.
\end{problem}
\begin{solution}
	The first part is an immediate consequence of Poincare's theorem (a previous problem in this section). If you don't want to invoke the theorem, you can prove this statement in the same fashion.
	
	For the second part, consider the Sylow 5-groups of a group $G$ of order 120. There can either be 1 or 6 of them. In the former case, we have a normal subgroup so $G$ is not simple. In the latter, $G$ acts on the Sylow 5-groups by conjugation, producing a map $G \to S_6$. If this map is not injective, then its kernel is a proper normal subgroup. If the image strictly contains $A_6$, then the preimage of $A_6$ is a normal subgroup. Otherwise we have an embedding of $G$ within $A_6$. But $A_6$ has no index 3 subgroups, so this is impossible.
\end{solution}