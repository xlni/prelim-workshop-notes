\section{7/23/18: Rings 1}
\begin{problem}
	Let $F$ be a field and let $M_n(F)$ be the ring of $n\times n$ matrices with coefficients in $F$. Prove that $M_n(F)$ has no nontrivial (two-sided) ideals. What can you conclude about ring homomorphisms from $M_n(F)$?
\end{problem}
\begin{solution}
	Suppose $\mf{a}$ is a nonzero ideal, and let $A\in \mf{a}$. Multiply $A$ (on both sides as needed) by matrices which are entirely zero except for a single 1 in one spot in order to get a matrix with only one non-zero entry. Then use permutation matrices (again on both sides as needed) to move that non-zero entry to all diagonal locations---take the sum and scale to get the identity. Thus $\mf{a} = (1)$.
	
	It follows that ring homomorphisms from $M_n(F)$ are always injective (provided that the target is not the zero ring).
	
	We remark that the statement of the problem is certainly untrue if we consider one-sided ideals: consider the ideal of matrices whose kernels contain a fixed subspace of $F^n$, for example (or whose images are contained within a fixed subspace).
\end{solution}

\begin{problem}
	Let $R$ be the set of complex numbers of the form
	\[
		a + 3bi, \quad a,b\in \bb{Z}.
	\]
	Prove that $R$ is a subring of $\bb{C}$ and that $R$ is an integral domain but not a unique factorization domain.
\end{problem}
\begin{solution}
	To see that it is a subring of $\bb{C}$, one needs only check closure, which we omit. The fact that $R$ is an integral domain is obvious, because $\bb{C}$ is one (indeed, it is a field).
	
	To show that $R$ is \emph{not} a UFD, as in the words of my former algebra professor Andrei Negut, ``use the norm!''. (The ``norm'' we use is actually the ``norm squared'' but it's more convenient because it's an integer and still enjoys the multiplicative properties of the norm.)
	\[
		N(a+3bi) \coloneqq a^2 + 9b^2.
	\]
	The possible values of $N$ are $0,1,4,9,\ldots$. Now consider
	\[
		(1 + 3i)(1 - 3i) = 10 = 2 \cdot 5.
	\]
	If $R$ were a UFD, since $N(2) = 4$ is irreducible, it must divide one of the two terms on the left. This is because we have ``unique factorization into irreducibles, up to units,'' or alternatively that irreducibles are prime in a UFD. But norm considerations show that this is impossible.
\end{solution}

\begin{problem}
	Let $m$ and $n$ be positive integers. Prove that the ideal generated by $x^m - 1$ and $x^n - 1$ in $\bb{Z}[x]$ is principal.
\end{problem}
\begin{rem*}
	The division algorithm works in $R[x]$ provided the thing we're dividing by is monic (leading coefficient is a unit).
\end{rem*}
\begin{solution}
	Use the Euclidean algorithm to show that $x^{\gcd(m,n)} - 1 \in (x^m-1, x^n -1)$. Then we conclude
	\[
		(x^{\gcd(m,n)}-1) = (x^m-1, x^n-1)
	\]
	because the other inclusion is obvious by divisibility.
\end{solution}

\begin{problem}
	Let $R$ be a finite commutative ring with unity which has no zero-divisors and contains at least one element other than $0$. Prove that $R$ is a field.
\end{problem}
\begin{solution}
	We know that $R$ is not the zero ring, so to show that $R$ is a field, we need only show that every element of $R$ is a unit (has an inverse). To that end, let $r\in R$ and consider the map ``multiplication by $r$.'' Since $R$ has no zero-divisors, we deduce
	\[
		-\cdot r \colon R\to R
	\]
	is injective. But $R$ is \emph{finite}, so that means it's also surjective. Hence $r$ has an inverse, as desired.
	
	The statement is obviously untrue without the finiteness assumption; consider $R = \bb{Z}$.
\end{solution}

\begin{problem}
	Let $F$ be a field and $X$ a finite set. Let $R(X,F)$ be the ring of all functions from $X$ to $F$, endowed with pointwise operations. What are the maximal ideals of $R(X,F)$?
\end{problem}
\begin{solution}
	There are a number of ways that one could show the ideals of $R(X,F)$ are precisely sets of the form $I(S)$ where $S\subeq X$ and $I$ denotes ``ideal vanishing on.'' Possibly the most direct and elementary way is by considering ``supports'' and then using multiplication by indicator functions.
	
	Alternatively, one can use, in some form, $\Spec \prod = \coprod \Spec$ for finite products. The ideals of a product are products of ideals, and the prime ideals are a prime ideal in one of the rings, times all the other rings.
\end{solution}

\begin{problem}
	Let $R$ be a principal ideal domain and let $I$ and $J$ be nonzero ideals. Show that $IJ = I \cap J$ if and only if $I+J = R$.
\end{problem}
\begin{solution}
	The implication $I+J = R \implies IJ = I \cap J$ does not require that $R$ is a PID. Just note that $i+j = 1$ for some $i\in I$ and $j\in J$, and then for $x\in I\cap J$ we have
	\[
		x = x(i+j) = xi + xj \in IJ.
	\]
	Of course, it is always true that $IJ \subeq I\cap J$.
	
	For the other direction, we proceed more or less ``directly.'' I had some trouble with this, mostly because I didn't ``push aggressively enough'' with the information provided. Write $I+J = (k)$ for some $k\in R$, and $I = (m)$ and $J = (n)$. So then we must have $m = kx$ and $n = ky$ for some $x,y \in R$. Now $kxy \in I \cap J \subeq IJ$, meaning that it is a multiple of $(kx)(ky)$. Since we are working in a domain, it follows that $1$ is a multiple of $k$, i.e. that $k$ is a unit. so $I+J = R$.
	
	There are some analogies with this argument to the intuitive lcm and gcd argument for $\bb{Z}$.
\end{solution}

\begin{problem}
	By the fundamental theorem of algebra, the polynomial $x^3 + 2x^2 + 7x + 1$ has three complex roots, $\alpha_1, \alpha_2$ and $\alpha_3$. Compute $\alpha_1^3 + \alpha_2^3 + \alpha_3^3$.
\end{problem}
\begin{solution}
	When I saw this problem, I knew to write the desired expression in terms of elementary symmetric functions and to use Vieta's formulas. What I did \emph{not} think of, however, is that I can first simplify the desired expression by using the fact that the $\alpha_i$ are roots of the given polynomial!
	
	That makes the problem somewhat simpler (although the original approach would've worked too). The simplification would be a lot more important if the desired expression had very high degree.
\end{solution}

\begin{problem}
	Let $\mf{a}$ be the ideal in $\bb{Z}[x]$ generated by $5$ and $x^3 + x + 1$. Is $\mf{a}$ prime?
\end{problem}
\begin{solution}
	Let $R = \bb{Z}[x]$. We are interested in whether $R / \mf{a}$ is an integral domain. We will use the fact that
	\[
		R/\mf{a} = \frac{R/(5)}{\mf{a}/(5)} = (\bb{Z}/5)[x]/(x^3 + x + 1).
	\]
	$\bb{Z}/5$ is a field, so $(\bb{Z}/5)[x]$ is a UFD and our question is equivalent to asking whether $x^3 + x + 1$ is irreducible. It doesn't have a root in $\bb{Z}/5$, so it is.
\end{solution}

\begin{rem*}
	There's a potentially useful ``converse'' of sorts; see Theorem 2.28 in Altman-Kleiman's CA notes.
\end{rem*}

\begin{problem}
	Let $f_n(x) = x^{n-1} + x^{n-2} + \cdots + x + 1$. Show that $f_n(x)$ is irreducible in $\bb{Q}[x]$ if $n$ is prime. What if $n$ is composite?
\end{problem}
\begin{solution}
	Apply the shift $x = (y+1)$, noting that
	\[
		f_n(x) = \frac{x^n - 1}{x - 1} = ((y+1)^n - 1)/y
	\]
	satisfies Eisenstein's criterion for irreducibility. So the original (unshifted) polynomial is irreducible too.
	
	If $n = mk$ is composite, then
	\[
		f_n(x) = (x^{m-1} + \cdots + 1)(x^{(k-1)m} + x^{(k-2) m} + \cdots + 1).
	\]
	I don't know if there's a motivated way of seeing this.
\end{solution}

\begin{rem*}
	Good idea to review cyclotomic polynomials, perhaps? Maybe that can offer some motivation.
\end{rem*}

\begin{problem}
	Factor $x^4 + x^3 + x + 3$ completely in $(\bb{Z}/5)[x]$.
\end{problem}
\begin{solution}
	It has no roots in $\bb{Z}/5$, so if it did factor, it must factor into two quadratics. Since $\bb{Z}/5$ is a field, we can assume that the two quadratics are monic:
	\[
		(x^2 + ax + b)(x^2 + cx + d) = x^4 + (a+c) x^3 + (d + ac + b) x^2 + (ad + bc) x + bd.
	\]
	Is there anything better to do here than trial and error? We can eliminate a few possibilities since the quadratics need to be irreducible, but still...
	
	There are 10 irreducible quadratics over $\bb{Z}/5$. They are
	\begin{gather*}
		x^2 + 2, x^2 + 3, x^2 + x + 1, x^2 + x + 2,\\
		x^2 + 2x + 3, x^2 + 2x + 4, x^2 + 3x + 3, x^2 + 3x + 4,\\
		x^2 + 4x + 1, x^2 + 4x + 2.
	\end{gather*}
	There's no way to factor it.
\end{solution}
