\section{8/8/18: Complex analysis 2}
S\&S has the following example of a function which is holomorphic on $\bb{D}$, can be extended continuously to $\overline{\bb{D}}$, but cannot be extended holomorphically past:
\[
	\sum_{n=0}^\infty 2^{-n\alpha}z^{2^n}
\]
for some $0<\alpha<\infty$. Trying to \emph{prove} this, however, is not that easy (I haven't done it). It's a particularly pathological example involving a nowhere-differentiable function.

There are three kinds of actual singularities: removable, pole, and essential. They can be distinguished using the Laurent series about the singularity: consider what happens to behavior around the point after multiplying the function by $(z-z_0)^n$. If using $n=1$ causes the function to approach 0 at $z_0$, then it's removable. If this is achieved by some higher power, then it's a pole (if $n=2$, then we have a simple pole, corresponding to the $a_{-1}/(z-z_0)$ term being the lowest in the series expansion). If this is not achieved by any value of $n$, then the singularity is essential.

There's another kind of ``singularity'' to keep in mind: branch points. Basically, some functions can't be defined globally (for topological reasons or otherwise), or more specifically, in a neighborhood around a specific point. Consider $f(z) = \sqrt{z}$, which cannot be defined in a neighborhood of the origin. Such examples can be useful for exhibiting holomorphic functions that can't be extended in certain ways.

Identity theorem: if a holomorphic function vanishes on a domain with an accumulation point, then it is identically zero.

Schwarz reflection principle: often used in tandem with the preceding. If $f(z)$ is holomorphic, so is $\overline f(\overline{z})$. That's the trivial form: a less trivial form states that it's possible to extend a holomorphic function across the real axis if it takes real values on the real axis. The subtlety is holomorphicity along the axis, but that can be shown using Morera's theorem.

Generalized Liouville's theorem: if there exists $c$ such that $|f(z)| \leq c|z|^n$ for sufficiently large $|z|$, then $f$ is a polynomial of degree at most $n$. Proved using induction, with ordinary Liouville's as base case.

Got a pole at infinity? Consider the function $f(1/z)$ instead to get a pole at zero.

Laurent expansion is a powerful tool that is worth trying, especially for analyzing singularities.

For an essential singularity, for \emph{any} complex number $w$, it is possible to pick a sequence converging to the singularity whose image under $f$ converges to $w$. Little/great Picard's theorems. Little: an entire and non-constant function misses at most one point. Great: on any punctured neighborhood of an essential singularity, the image of $f$ misses at most one point. Great can be used to prove little by considering polynomial and non-polynomial cases separately.

Rouch\'e's theorem: a useful tool for counting roots with multiplicity. Might as well remember the symmetric version: if
\[
	|f(z)-g(z)| < |f(z)| + |g(z)|
\]
everywhere on the simple contour $\partial K$ for some bounded region $K$, then $f,g$ have the same number of roots counting multiplicity inside $K$. Proof: homotopy invariance of winding number (which counts number of roots with multiplicity).

If you want to define the logarithm of a function, consider the fact that ``$\ln f$'' if it exists would differentiate to $f'/f$. A holomorphic function has a primitive on a simply connected set.