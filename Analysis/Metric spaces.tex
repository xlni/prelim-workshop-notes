\section{7/20/18: Metric spaces}
\begin{problem}
	Let $X \subeq \bb{R}^n$ be compact and let $f\colon X\to \bb{R}$ be continuous. Given $\epsilon > 0$, show there is an $M$ such that for all $x,y\in X$,
	\[
		|f(x) - f(y)|\leq M|x-y|+\epsilon.
	\]
\end{problem}
\begin{solution}
	A continuous function on a compact set is uniformly continuous, and also is bounded (by some $C$). So there exists $\delta > 0$ for which $|x-y| < \delta \implies |f(x)-f(y)|<\epsilon$.
	
	Then pick $M$ large enough so that $M\delta > 2C$. 
\end{solution}

\begin{problem}
	Let $K$ be a continuous function on the closed unit square satisfying $|K(x,y)|<1$ for all $x$ and $y$. Show that there is a continuous function $f(x)$ on $[0,1]$ such that we have
	\[
		f(x) + \int_0^1 K(x,y)f(y) \mathrm{d} y = e^{x^2}.
	\]
	Can there be more than one such function $f$?
\end{problem}
\begin{solution}
	The fact that $|K(x,y)|<1$ is a hint that the fixed point theorem for a contraction may be involved. According to Albert, this theorem is a common way of answering these integral equation type problems.
	
	Consider the operator $T$ defined as
	\[
		T(f)(x) = e^{x^2} - \int_0^1 K(x,y)f(y) \mathrm{d} y.
	\]
	We claim that $T\colon C([0,1]) \to C([0,1])$ is a contraction, where $C([0,1])$ is equipped with the supremum norm. There are two parts to this claim: the fact that $T(f)$ is a continuous function of $x$, and that $T$ is a contraction. Both parts essentially rely on compactness of the unit square, which tells us that
	\begin{itemize}
		\item $|K(x,y)|$ is uniformly continuous. This can be used to show that $T(f)$ is continuous.
		\item $|K(x,y)|$ has a maximum strictly less than $1$. This is used to show that $T$ is a contraction. Note that the definition of a contraction is \emph{not} that $d(T(f),T(g)) < d(f,g)$, but that there is a constant $\alpha <1$ such that $d(T(f),T(g)) \leq \alpha d(f,g)$.
	\end{itemize}
	Finally, we use the fact that $C([0,1])$ is a complete metric space. This can probably just be cited, but the proof is not difficult: it just uses the fact that if continuous functions converge uniformly to another function, then that other function is also continuous.
	
	With all these ingredients, we can invoke the fixed point theorem to deduce that $T$ has a \emph{unique} fixed point which is then the solution to the integral equation.
\end{solution}

\begin{lem}\label{lem:separate-compact-closed-in-metric}
	In a metric space, a compact set and a closed set can be separated by a positive distance.
\end{lem}

\begin{problem}
	Let $X$ be a compact metric space and $f\colon X\to X$ an isometry. Show that $f(X)=X$.
\end{problem}
\begin{solution}
	Assume otherwise, and chase an element $x\in X\setminus f(X)$ around using $f$. Note that $x$ has some positive distance away from $f(X)$, which is compact.
\end{solution}

\begin{problem}
	Let $F$ be a uniformly bounded, equicontinuous family of real valued function on the metric space $(X,d)$. Prove that the function
	\[
		g(x) = \sup\{f(x) : f\in F\}
	\]
	is continuous.
\end{problem}
\begin{solution}
	The hypotheses appearing in this question suggest the usage of Arzela-Ascoli, but that is actually a red herring. This can be easily proved directly by noting that if $x, \epsilon > 0$ are given and $\delta$ is picked as in the definition of equicontinuity, then for $|x-y| < \delta$ we have
	\begin{align*}
		g(y) &= \sup_f f(y) \leq \sup_f (f(x) + \epsilon) = \sup_f f(x) + \epsilon = g(x) + \epsilon\\
		g(y) &= \sup_f f(y) \geq \sup_f (f(x) - \epsilon) = \sup_f f(x) - \epsilon = g(x) - \epsilon.
	\end{align*}
\end{solution}

\begin{problem}
	Let $X \subeq \bb{R}^n$ be a closed set and $r$ a fixed positive real number. Let $Y = \{y\in \bb{R}^n : \exists x\in X, |x-y| = r\}$. Show that $Y$ is closed.
\end{problem}
\begin{solution}
	Show that for a fixed point $p\in \bb{R}^n$, the function $d(p,-)$ is closed, using Lemma~\ref{lem:separate-compact-closed-in-metric} and compactness of the sphere. If $p \notin Y$, then $d(p,X)$ is closed and does not contain $r$. So pick a ball around $r \in \bb{R}$ which doesn't meet $d(p,X)$, and consider the ball around $p \in \bb{R}^n$ of the same radius. This ball will be disjoint from $Y$ by the triangle inequality. Thus $Y$ is closed.
	
	Well, you could reorganize this argument and just show that the sphere of radius $r$ around $p$ is separated by some $\epsilon > 0$ from $X$.
\end{solution}