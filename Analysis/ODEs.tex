\section{7/16/18: Ordinary differential equations}

\begin{problem}[Gronwall's inequality]
	Suppose $f$ is a differentiable function from the reals into the reals. Suppose $f'(x) > f(x)$ for all $x\in \bb{R}$, and $f(x_0) = 0$. Prove that $f(x) > 0$ for all $x > x_0$.
\end{problem}
\begin{solution}[1]
	The direct proof that first came to my mind goes as follows: consider $f^{-1}\{0\} \cap (x_0, \infty)$. Show that this set has a minimum; call it $x_1$.
	
	Then use the mean value theorem to deduce that $f'$ vanishes between $x_0$ and $x_1$. But then $f$ must be negative there, and from here we can derive a contradiction.
\end{solution}
\begin{solution}[2]
	This is the solution Albert gave in class. Basically, rearrange the inequality as
	\[
		f'(x) - f(x) > 0
	\]
	and then introduce an ``integrating factor'' so that the LHS is the derivative of a product:
	\begin{align*}
		e^{-x} f'(x) - e^{-x} f(x) &> 0\\
		\deriv[x](e^{-x}f(x)) &> 0. 
	\end{align*}
	Observe that $e^{-x}f(x)$ is positive and strictly increasing for $x > x_0$. The same is true of the function $e^x$, so we deduce their product $f(x)$ is positive and strictly increasing as well for $x > x_0$.
\end{solution}
\begin{rem*}
	If one has an inequality like $u'(t) < \beta(t) u(t)$, one first finds $v$ which solves $v'(t) = \beta(t) v(t)$ and then computes the derivative of $u/v$ to deduce that this ratio is increasing or decreasing (as the case may be). Then compare to its initial value at $t_0$.
\end{rem*}

\begin{problem}
	Let $n$ be an integer larger than $1$. Is there a differentiable function on $[0,\infty)$ whose derivative equals its $n$th power and whose value at the origin is positive?
\end{problem}
\begin{solution}
	The differential equation is
	\[
		\deriv[x][y] = y^n; \quad y(0) > 0.
	\]
	The conditions of Picard's theorem are satisfied so we have local existence and uniqueness. The latter is the useful deduction---the above differential equation is easy to solve directly. Simply observe that the solutions all have vertical asymptotes at positive $x$, so there can be no continuous solution defined on all of $[0,\infty)$.
\end{solution}

\begin{problem}
	Prove that the initial value problem
	\[
		\deriv[t][x] = 3x + 85 \cos x; \quad x(0)=77
	\]
	has a solution $x(t)$ defined for all $t\in \bb{R}$.
\end{problem}
\begin{solution}
	For this problem, it is very useful to have a quantitative form of Picard's theorem. Suppose we have a differential equation of the form
	\[
		\deriv[t][x] = f(x,t),
	\]
	i.e. a ``slope field'' given by $f$, defined on some rectangle $[t_0 - a, t_0 + a] \times [x_0 - b, x_0 + b]$.
	
	If $f$ is continuous with respect to $t$ and uniformly Lipschitz with respect to $x$ (in the sense that the Lipschitz constant may be picked independently of $t$) then there is a unique solution passing through $(t_0, x_0)$ defined for $t\in [t_0 - \alpha, t_0 + \alpha]$ where
	\[
		\alpha = \min(a, b/(\sup |f|)).
	\]
	In our problem, by picking $b$ large enough, we can always ensure that $b/\sup f > 1/4$ for example. So repeated application of Picard's theorem shows that there is a (unique) solution $x(t)$ defined for all $t \in \bb{R}$.
\end{solution}

\begin{problem}
	Let $f\colon \bb{R}\to\bb{R}$ be a continuous nowhere vanishing function, and consider the differential equation
	\[
		\deriv[x][y] = f(y).
	\]
	\begin{enumerate}
		\item For each real number $c$, show that this equation has a unique continuously differentiable solution $y=y(x)$ on a neighborhood of $0$ which satisfies the initial condition $y(0) = c$.
		\item Deduce the conditions on $f$ under which the solution $y$ exists for all $x\in \bb{R}$, for every initial value $c$.
	\end{enumerate}
\end{problem}
\begin{solution}\hfill
	\begin{enumerate}
		\item The idea is that we can just ``solve'' this differential equation by ``separation of variables.'' To be rigorous, we will invoke the inverse function theorem.
		
		Let us instead consider the differential equation
		\[
			\deriv[y][x] = \frac{1}{f(y)}.
		\]
		The RHS is a continuous function; consider $x(y) = \int_c^y \frac{\mathrm{d}t}{f(t)}$. This function is continuously differentiable, and is the \emph{unique} solution to the above equation with $x(c)=0$. Moreover, $x'(c) \neq 0$ so the inverse function theorem tells us that it (locally) has a continuously differentiable inverse $y(x)$ which satisfies $y(0) = c$ and the differential equation
		\[
			\deriv[x][y] = f(y)
		\]
		as desired. Uniqueness also follows from the IFT.
		\item The function $x(y)$ defined previously is either strictly increasing or strictly decreasing. In order for $y(x)$ to be defined for all $x\in \bb{R}$, it is necessary and sufficient for $x(y)$ to be surjective.
	\end{enumerate}
\end{solution}

\begin{problem}
	Consider the equation
	\[
		\deriv[x][y] = y -\sin y.
	\]
	Show that there is an $\epsilon >0$ such that if $|y_0|<\epsilon$, then the solution $y = f(x)$ with $f(0) = y_0$ satisfies
	\[
		\lim_{x\to -\infty} f(x) = 0.
	\]
\end{problem}
\begin{solution}
	Actually, $\epsilon$ can be taken to be anything---the result holds for any $y_0$.
	
	Albert mentioned that there isn't a specific tool or theorem that immediately knocks out this problem. The strategy here is more or less to slowly chip away at the problem. Drawing a picture helps \emph{a lot}.
	
	Note that $y - \sin y$ is an odd function. If $y=f(x)$ is a solution to the differential equation, then so is $y = -f(x)$. Also, the hypotheses of Picard's theorem are satisfied, so solutions are locally unique. The function $y=0$ is the solution with $y_0 = 0$.
	
	So we may as well assume $y_0 > 0$. The solution $y$ cannot cross the $x$-axis, as that would violate uniqueness of solutions. Thus it is always positive, and the differential equation also tells us that it is strictly increasing.
	
	Hence the limit in question is the infimum of the range of $y$. We have already established that 0 is a lower bound. Moreover, if $L > 0$ is a lower bound, then $y'$ is \emph{also} bounded below thanks to the differential equation. But then, by the mean value theorem for instance, we must have that $y$ crosses the $x$-axis, a contradiction. Therefore 0 is indeed the desired infimum and we are done.
\end{solution}