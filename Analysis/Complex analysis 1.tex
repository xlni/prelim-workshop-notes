\section{8/3/18: Complex analysis 1}
Morera's theorem is a useful way of proving that something is holomorphic when you aren't given much else about the function. Holomorphic functions enjoy many properties that real differentiable functions do not: for example the uniform limit of holomorphic functions is also holomorphic, while the analogous statement for real functions is blatantly false by Stone-Weierstrass (any continuous function on a closed interval can be approximated uniformly by polynomials).

By the way, for Morera's theorem, you need only show that the integral is zero on all elements of a family $\mathcal{F}$ of toy contours which is closed under translation and dilation. So for example, the family of all circles would suffice.

If a holomorphic function is bounded near a singularity, then that singularity is removable. A singularity $a$ is removable, by definition, when $\lim_{z\to a} (z-a)f(z) = 0$.

Open mapping theorem: If $U$ is a domain (connected open subset) and $f\colon U\to \bb{C}$ is holomorphic and \emph{non-constant}, then it is open. Contrast this with the real case (consider a parabola for example).

By the way, that means that if you have a non-constant holomorphic map from an open set to any set $C\subset\bb{C}$, the image of the map lands in the interior of $C$. This is useful if $C$ is the closed unit disk for example, because then we can use many of the below tools on the open unit disk.

Conformal map: holomorphic with everywhere nonzero derivative. Note that this is a local condition: preserving angles. Some people use conformal to mean ``one-to-one and holomorphic,'' which is stronger and involves a global condition that is not satisfied by e.g. the exponential function. Why is this stronger? One-to-one and holomorphic implies nonzero derivative everywhere. This can proved by contradiction using series expansion and Rouch\'e's theorem.

Schwarz lemma: should be considered for just about any problem about the open unit disk $\bb{D}$. Let $f\colon \bb{D} \to \bb{D}$ be holomorphic and $f(0)=0$. Then $|f(z)| \leq |z|$ and $|f'(0)| \leq 1$. Moreover, if equality holds for either of the preceding inequalities for \emph{some} value of $z$, then $f$ is in fact a rotation.

If you're asked to find the maximum possible value of some holomorphic function at a given point given some constraints on the function, this lemma may be useful.

Blaschke factor: often used in tandem with the preceding. Special function with nice properties:
\begin{itemize}
	\item $f_{z_0} \colon \bb{D} \to \bb{D}$ is an automorphism. In fact, it is its own inverse.
	\item $f_{z_0}(z_0) = 0$.
\end{itemize}
The formula is given by
\[
	f_{z_0}(z)=\frac{z-z_0}{\overline{z_0} z - 1}, \quad z_0 \in \bb{D}
\]
though often times just knowing such a function exists is enough to solve problems.

Mobius transformation: useful for mapping between half-plane and unit disk. We construct a map from the upper half-plane to the unit disk by
\[
	\frac{z-i}{z+i}
\]
since points in the upper half-plane are closer to $i$ than to $-i$. There is a map $GL(2,\bb{C}) \to Aut(\bb{\widehat{\bb{C}}})$
which sends
\[
	\begin{bmatrix}
	a & b\\
	c & d
	\end{bmatrix} \mapsto \frac{az + b}{cz + d}
\]
that is, in fact, a group homomorphism. In particular, this gives a convenient way of composing/inverting Mobius transformations.

Maximum modulus principle: if $f$ is holomorphic at $z_0$, then we cannot have $|f(z_0)| \geq |f(z)|$ for all $z$ in any neighborhood of $z_0$. This can be proven easily using the open mapping theorem. Minimum modulus is fine too, provided it's nonzero. Also useful to state as: a function obtains its maximal modulus along boundary (of closed bounded domain).

Liouville's theorem: an entire bounded function is constant. Proved by using Cauchy's integral formula for the derivative along arbitrarily large circles around any given point.

If you're given constraints on the real part of a function, consider exponentiation. Or if you have constraints on the imaginary part, first multiply by $i$ and then exponentiate.