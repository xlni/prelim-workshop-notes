\usepackage[latin1]{inputenc}
\usepackage{amsmath}
\usepackage{amsthm}
\usepackage{mathtools}
\usepackage{amsfonts}
\usepackage{amssymb}
%\usepackage{graphicx}
\usepackage[headsep=0.15in, left=0.5in, right=0.5in, top=0.6in, bottom=0.5in]{geometry}
\usepackage[textlf,mathlf]{MinionPro}
\usepackage{fancyhdr}
\usepackage{tikz-cd}
\usepackage{xparse} %for multiple optional arguments

%Environments
\numberwithin{equation}{section}
\numberwithin{figure}{section}
\theoremstyle{definition}
\newtheorem{example}{\protect\examplename}[section]
\theoremstyle{remark}
\newtheorem*{rem*}{\protect\remarkname}
\theoremstyle{plain}
\newtheorem{thm}{\protect\theoremname}[section]
\theoremstyle{plain}
\newtheorem{cor}{\protect\corollaryname}[section]
\theoremstyle{definition}
\newtheorem{defn}{\protect\definitionname}[section]
\theoremstyle{plain}
\newtheorem{prop}{\protect\propositionname}[section]
\theoremstyle{definition}
\newtheorem{problem}{\protect\problemname}[section]
\theoremstyle{plain}
\newtheorem{lem}{\protect\lemmaname}[section]
\theoremstyle{definition}
\newtheorem{xca}{\protect\exercisename}[section]
\newenvironment{solution}[1][\unskip]{\paragraph{Solution {#1}:}}{\hfill$\square$}

\providecommand{\definitionname}{Definition}
\providecommand{\examplename}{Example}
\providecommand{\lemmaname}{Lemma}
\providecommand{\problemname}{Problem}
\providecommand{\propositionname}{Proposition}
\providecommand{\remarkname}{Remark}
\providecommand{\corollaryname}{Corollary}
\providecommand{\theoremname}{Theorem}
\providecommand{\exercisename}{Exercise}

%Fancy header and footer configuration
\fancyhf{}
\pagestyle{fancy}
\makeatletter
\let\ps@plain\ps@fancy   % Plain page style = fancy page style
\makeatother

%macros, etc.
\newcommand{\bb}[1]{\mathbb{#1}}
\newcommand{\subeq}{\subseteq}
\DeclareMathOperator{\Stab}{Stab}
\NewDocumentCommand{\deriv}{O{x} O{}}{\frac{\mathrm{d}{#2}}{\mathrm{d}{#1}}}